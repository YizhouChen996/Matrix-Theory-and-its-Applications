\documentclass{article}
\usepackage{blindtext}
\usepackage[utf8]{inputenc}
\usepackage{amsmath,bm}
\usepackage{amstext}
\usepackage{amsfonts}
\usepackage{amsmath}
\usepackage{multirow}
\usepackage{enumerate}
\usepackage{xeCJK}
\setCJKmainfont{STKaiti}
\usepackage{algorithm}
\usepackage{algorithmic}
\renewcommand{\algorithmicrequire}{ \textbf{输入:}} %Use Input in the format of Algorithm
\renewcommand{\algorithmicensure}{ \textbf{输出:}} %UseOutput in the format of Algorithm
\usepackage{graphicx}

\title{矩阵论及其应用\\习题2}
\author{陈轶洲 MF20330010}
\begin{document}
	\maketitle
	\numberwithin{equation}{section}

\section{}	
令$ A = \begin{bmatrix}1 & -1 \\ -1 & 3 \end{bmatrix} $,则$ (u,v)=u^TAv $满足:
\begin{enumerate}[1)]
	\item 
	$(v, u)=y_1x_1-y_1x_2-y_2x_1+3y_2x_2=\begin{bmatrix}x_1 & x_2\end{bmatrix}\begin{bmatrix}y_1-y_2\\ -y_1+3y_2 \end{bmatrix}  =u^TAv=(u,v)$
	\item $(ku,v)=(ku)^TAv=k(u^TAv)=k(u,v)$ 
	\item $(u+w,v)=(u+w)^TAv=u^TAv+w^TAv=(u,v)+(w,v)$
	\item $(u,u) = u^TAu=x_1^2-2x_1x_2+3x_2^2=(x_1-x_2)^2+2x_2^2\ge 0$且仅当$ u=0 $时等号成立
\end{enumerate}
综上,$ (u,v) $是$R^2$上的内积。

\section{}
若$ (\alpha,\beta)_1,(\alpha,\beta)_2 $是$ V $的不同内积,则:
\begin{enumerate}[1)]
	\item $ (\alpha,\beta)=(\alpha,\beta)_1+(\alpha,\beta)_2 = (\beta,\alpha)_1+(\beta,\alpha)_2 =(\beta,\alpha) $
	\item $ (k\alpha,\beta)=(k\alpha,\beta)_1+(k\alpha,\beta)_2 = k(\alpha,\beta)_1+k(\alpha,\beta)_2 =k(\alpha,\beta) $
	\item $ (\alpha+\delta,\beta)=(\alpha+\delta,\beta)_1+(\alpha+\delta,\beta)_2 = (\alpha,\beta)_1+(\alpha,\beta)_2 +(\delta,\beta)_1+(\delta,\beta)_2=(\alpha,\beta)+(\delta,\beta) $
	\item $ (\alpha,\alpha) =(\alpha,\alpha)_1+(\alpha,\alpha)_2\ge 0 $当且仅当$ \alpha=0 $时等号成立
\end{enumerate}
设$ (\alpha,\beta)_1 $为$ V $上的内积,定义$ (\alpha,\beta)=(\alpha,\beta)_1+(\beta,\alpha)_1 $,下面证明$ (\alpha,\beta) $为$ V $上的内积:
\begin{enumerate}[1)]
	\item $ (\alpha,\beta)=(\alpha,\beta)_1+(\beta,\alpha)_1 = (\beta,\alpha)_1+(\alpha,\beta)_1 =(\beta,\alpha) $
	\item $ (k\alpha,\beta)=(k\alpha,\beta)_1+(\beta,k\alpha)_1 = k(\alpha,\beta)_1+k(\beta,\alpha)_1 =k(\alpha,\beta) $
	\item $ (\alpha+\delta,\beta)=(\alpha+\delta,\beta)_1+(\beta,\alpha+\delta)_1 = (\alpha,\beta)_1+(\beta,\alpha)_1 +(\delta,\beta)_1+(\beta,\delta)_1=(\alpha,\beta)+(\delta,\beta) $
	\item $ (\alpha,\alpha) =2(\alpha,\alpha)_1\ge 0 $当且仅当$ \alpha=0 $时等号成立
\end{enumerate}
证毕。

\section{}
令$ A=[\alpha_1,\alpha_2,\dots,\alpha_n]$, $B=[\beta_1,\beta_2,\dots,\beta_n] $,由题意知A的列向量是B中列向量的线性组合,故必存在实数矩阵$ C\in R^{n\times n} $使得$ A=BC $成立。\\
于是等式两边左乘B得到:$ BA=B^2C $。又已知$B^2=B$,故$ BA=BC=A $,证毕。

\section{}
使用Schmidt方法构造正交基与标准正交基:
\begin{equation}
	\begin{aligned}
		q'_1 &= v_1\quad q_1=\frac{q'_1}{||q'_1||}=\begin{bmatrix}0\\\frac{2}{\sqrt{6}}\\\frac{1}{\sqrt{6}}\\\frac{1}{\sqrt{6}}\end{bmatrix}\\
		q'_2 &=v_2-(v_2,q_1)q_1=\begin{bmatrix}0\\\frac{1}{3}\\\frac{2}{3}\\\frac{2}{3}\end{bmatrix}\quad q_2 =\frac{q'_2}{||q'_2||}=\begin{bmatrix}0\\\frac{1}{3}\\\frac{2}{3}\\\frac{2}{3}\end{bmatrix}\\
		q'_3 &=v_3-(v_3,q_1)q_1-(v_3,q_2)q_2=\begin{bmatrix}1\\-\frac{1}{3}\\-\frac{1}{6}\\-\frac{7}{6}\end{bmatrix}\quad q_3 =\frac{q'_3}{||q'_3||}=\frac{2}{\sqrt{10}}\begin{bmatrix}1\\-\frac{1}{3}\\-\frac{1}{6}\\-\frac{7}{6}\end{bmatrix}\\		
	\end{aligned}
\end{equation}

\end{document}
